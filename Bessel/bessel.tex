\documentclass{article}
\usepackage{mathtools}
\usepackage{pgf}
\usepackage{tikz}
\usepackage{mathdots}
\usetikzlibrary{arrows,automata,decorations.markings}
\makeatletter
\newcommand\mathcircled[1]{%
  \mathpalette\@mathcircled{#1}%
}
\newcommand\@mathcircled[2]{%
  \tikz[baseline=(math.base)] \node[draw,circle,inner sep=1pt] (math) {$\m@th#1#2$};%
}
\newcommand\myvec[1]{\vec{#1}\hspace{0.5mm}}
\makeatother

% TODO
% More motivation with trig and elliptic examples
% More on C
% Better notation for T, R and C
% Some blurb about f(x,y) = f(r,\theta) = f(z) = f(\myvec{p})

\begin{document}
\section{Introduction}
Many of the useful properties of the Fourier transform arise because it diagonalises translation.
More precisely, it expresses a function $f$ in the form
\[
f(x) = \frac{1}{\sqrt{2\pi}}\int_{-\infty}^\infty F(\omega)e^{i\omega x}d\omega
\]
i.e. as a kind of linear combination of functions $x\rightarrow e^{i\omega x}$.
These functions have the property that under translation they are multiplied by $i\omega$.
This gives a representation of functions that behaves particularly simply under translations.
It also gives a representation of functions that behaves well with a wide variety of other operations such as differentiation, integration and convolution.
These operations can be thought of as being conceptually built from translations.
For example differentiation is a bit like subtracting a function from its infinitesimal translate.

The $n$-dimensional Fourier transform gives a nice representation of functions in $n$ variables where we're interested in operations related to translations in $n$-dimensions.

The translations form a Lie group.
What about different groups like the rotations in 3D, $SO(3)$?
This group is non-commutative so we can't expect non-trivial diagonal representations.
But we can start with one axis, say the $z$-axis, and start by diagoinalising with respect to just the rotations around that axis.
If we use the usual coordinates $x$, $y$ and $z$ in 3D then we can write functions of $x$, $y$, and $z$ as functions of polar coordinates $r$, $\theta$ and $\phi$ where
\begin{align*}
x & = r\cos(\phi)\sin(\theta) \\
y & = r\sin(\phi)\sin(\theta) \\
z & = r\cos(\theta) \\
\end{align*}
For integer $m$, any function $f$ of the form $f(r,\theta,\phi) = g(r,\theta)e^{im\phi}$ is multiplied by a constant when rotated around the $z$-axis.
We can then seek to find functions $g$ that make $f$ as well behaved as possible when rotated around the other axes.
A good choice turns out to be the spherical harmonics.

Now consider another group: the Euclidean group $E(2)$ consisting of rotations and translations in the plane.
Can we find a family of functions that is well behaved under the action of this group?

\section{Partially diagonalising $E(2)$}
$E(2)$ is not commutative so we can't expect to completely diagonalise it.
Let's start by defining 2D polar coordinates by
\begin{align*}
x & = r\cos(\theta) \\
y & = r\sin(\theta) \\
\end{align*}
Now consider functions of the form
\begin{equation}
f_n(r,\theta)=b_n(r)e^{in\theta}. \label{rot}
\end{equation}
This mimics how the spherical harmonics work with $SO(3)$ and are clearly very well behaved with respect to rotations around the origin.
Write $R(\phi)$ for such a rotation by angle $\phi$.
Then 
\[
(R(\phi)f)(x,y) = f(x,y)e^{-im\phi}.
\]
So now we need to pick functions $b_m$ that behave nicely under translation.
We can make our life easier by focusing on infinitesimal translations.
So let's pick $b_n$ so that the $f_m$ take a particularly simple form when differentiated.
For convenience also define $z=x+iy$.

Differentiating and simplifying we get both
\begin{align}
\frac{\partial f_n}{\partial x} &= 
    \frac{e^{in\theta}}{r}\Big(-\frac{inyb_n(r)}{r}+xb_n'(r)\Big) \label{partialx}\\
\frac{\partial f_n}{\partial y} &= 
    \frac{e^{in\theta}}{r}\Big(\frac{inxb_n(r)}{r}+yb_n'(r)\Big) \label{partialy}
\end{align}
This is a little messy.
But there's a trick to simplify.
We have $x$ and $y$ appearing in different places in both of the right hand sides, but by forming $\partial f_n/\partial x+i\partial f_n/\partial y$ they both become multiples of $z$ which can be factored out.
We could do the same using $-i$ instead.
Writing
\begin{align*}
\frac{\partial}{\partial z} &= \frac{1}{2}\Big(\frac{\partial}{\partial x}-i\frac{\partial}{\partial y}\Big)\\
\frac{\partial}{\partial\bar{z}} &= \frac{1}{2}\Big(\frac{\partial}{\partial x}+i\frac{\partial}{\partial y}\Big)\\
\end{align*}
we get both
\begin{align*}
2\frac{\partial f_n}{\partial\bar{z}} & = e^{i(n+1)\theta}
    \Big(-\frac{nb_n(r)}{r}+b_n'(r)\Big)\\
2\frac{\partial f_n}{\partial z} & = e^{i(n-1)\theta}
    \Big(\frac{nb_n(r)}{r}+b_n'(r)\Big)
\end{align*}

We're interested in functions of the form
$f_n(r,\theta)=b_n(r)e^{in\theta}$
and these become particularly nice if we choose
\begin{align*}
b_{n+1}(r) & = \frac{nb_n(r)}{r}-b_n'(r) \\
b_{n-1}(r) & = \frac{nb_n(r)}{r}+b_n'(r)
\end{align*}
or equivalently
\begin{align}
\frac{2nb_n(r)}{r} & =  b_{n-1}(r)+b_{n+1}(r) \label{sum} \\
2b_n'(r) & =  b_{n-1}(r)-b_{n+1}(r) \label{deriv}
\end{align}
(Cf. AS-9.1.27)
Any family of functions $b_n$ satisfying these properties will result in the nice relations:
\begin{align*}
2\frac{\partial f_n}{\partial\bar{z}} & = -f_{n+1} \\
2\frac{\partial f_n}{\partial z} & = f_{n-1}.
\end{align*}
which are equivalent to
\begin{align}
2\frac{\partial f_n}{\partial x} & = f_{n-1}-f_{n+1} \label{sum3} \\
-2i\frac{\partial f_n}{\partial y} & = f_{n+1}+f_{n-1} \label{sum4}.
\end{align}

\section{Exponentiating up to translation}
Let's collect together the $b_i$ in a single generating function
\[
G(r,t) = \sum_{k=-\infty}^\infty b_k(r)t^k
\]
We get
\begin{align*}
\frac{\partial G(r,t)}{\partial r} &= \sum_{k=-\infty}^\infty \frac{1}{2}(b_{k-1}(r)-b_{k+1}(r))t^k\\
&= \sum_{k=-\infty}^\infty \frac{1}{2}(t-t^{-1})b_k(r)t^k \\
&= \frac{1}{2}(t-t^{-1})G(r,t)
\end{align*}
So we have
\[
G(r,t) = e^{\frac{1}{2}r(t-t^{-1})}g(t)
\]
for some $g$.
For simplicity we can pick $g(t) = 1$ and define $J_k(r)$ by
\begin{equation}
\boxed{\sum_{k=-\infty}^\infty J_k(r)t^k = G(r,t) = e^{\frac{1}{2}r(t-t^{-1})}}\label{defn}
\end{equation}
We can determine each of the $J_k(r)$ by expanding the Taylor series for $G(r, t)$.
The $J_k(r)$ are the Bessel functions of the first kind and Equation~(\ref{defn}) is often taken as their definition.
I hope it can be seen that the form of Equation~(\ref{defn}) comes directly from Equation~(\ref{sum3}) which in turn is what makes Equations~(\ref{partialx}) and (\ref{partialy}) as simple as possible.

Note that
\begin{align*}
\sum_{k=-\infty}^\infty (-1)^kJ_{-k}(r)t^k
&= \sum_{k=-\infty}^\infty J_{-k}(r)(-t)^k \\
&= \sum_{k=-\infty}^\infty J_k(r)(-t^{-1})^k \\
&= G(r,-t^{-1}) \\
&= G(r,t)
\end{align*}
meaning that $J_{-k}(r) = (-1)^kJ_{k}(r)$.

\section{An addition law}
We immediately get an addition law.
We can think of multiplication by $G(r,t)$ as corresponding to translation by $r$ so we expect an equation coming from performing a pair of translations in a row, i.e.
\[
G(r+s,t) = G(r,t)G(s,t)
\]
This also follows directly from the definition of $G$ as an exponential.
We get
\begin{align*}
G(r+s,t) &= \sum_{k=-\infty}^\infty J_k(r)t^k\sum_{l=-\infty}^\infty J_l(r)t^l\\
&= \sum_{k=-\infty}^\infty \Big(\sum_{m=-\infty}^\infty J_{m}(r)J_{k-m}(s) \Big) t^k
\end{align*}
and so
\[
J_k(r+s) = \sum_{m=-\infty}^\infty J_{m}(r)J_{k-m}(s)
\]
This is the well known addition law for Bessel functions, AS-???.
We appear to have succeeded in our quest.
But this is not the addition law we were looking for.
Our goal was to find addition laws corresponding to multiplication in the group $E(2)$.
But this addition law is a 1-dimensional addition law.
We need to go back to Equation~(\ref{rot}) and consider functions in the plane.
So define
\[
\boxed{B_n(r,\theta) = J_n(r)e^{in\theta}}.
\]
Because of Equation~(\ref{deriv}) we expect these to have nice behaviour under translation.
In fact, define
\begin{align*}
H(r,\theta,t) &= \sum_{k=-\infty}^\infty B_k(r,\theta)t^k \\
&= \sum_{k=-\infty}^\infty J_k(r)(te^{i\theta})^k \\
&= G(r,te^{i\theta}) \\
&= \exp\frac{1}{2}r(te^{i\theta}-t^{-1}e^{-i\theta}) \\
&= \exp\frac{1}{2}(tz-t^{-1}\bar{z}) \\
&= \exp\frac{1}{2}(x(t-t^{-1})+iy(t+t^{-1}))
\end{align*}
So if we have 2D vectors $\myvec{p}$ and $\myvec{q}$, then
\[
H(\myvec{p}+\myvec{q},t) = H(\myvec{p},t)H(\myvec{q},t)
\]
This tells us that
\begin{equation}
\boxed{B_n(\myvec{p}+\myvec{q}) = \sum_{k=-\infty}^\infty B_k(\myvec{p})B_{n-k}(\myvec{q})}\label{addition}
\end{equation}

Using Figure~\ref{triangle} we can write this in a more traditional, but also more obfuscated fashion.
Using the notation $\angle\myvec{p}$ to mean the angle between the vector $\myvec{p}$ and the $x$-axis:
\[
J_k(|\myvec{p}+\myvec{q}|)e^{ik\angle(\myvec{p}+\myvec{q})} = \sum_{m=-\infty}^\infty J_m(|\myvec{p}|)J_{k-m}(|\myvec{q}|)e^{im\angle\myvec{p}}e^{i(k-m)\angle\myvec{q}}
\]

\tikzset{->-/.style={decoration={
    markings,
    mark=at position #1 with {\arrow{>}}},postaction={decorate}}}

\begin{figure}
\centering
\begin{tikzpicture}[>=stealth]
\draw[->-=0.5] (0,0) node[left]{$0$}
%node[shift={(0.5,0.05)}]{$\theta$}
-- (4,-2) node[midway,below left]{$\myvec{p}$};
\draw[->-=0.5] (4,-2) node[shift={(-0.15,0.4)}]{$\phi$} -- (5,3) node[midway,right] {$\myvec{q}$};
\draw[->-=0.5] (0,0) -- (5,3) node[midway,above left] {$\myvec{p}+\myvec{q}$} node[shift={(-0.3,-0.4)}]{$\psi$};
\end{tikzpicture}
\caption{A triangle of vectors}
\label{triangle}
\end{figure}

We have
\begin{align*}
%\angle\myvec{p}+\theta & = \angle(\myvec{p}+\myvec{q}) \\
\angle\myvec{p}+\pi-\phi & = \angle(\myvec{q}) \\
\angle(\myvec{p}+\myvec{q})+\psi & = \angle(\myvec{q}) 
\end{align*}

Therefore
\[
J_k(|\myvec{p}+\myvec{q}|)e^{ik(\angle(\myvec{p}+\myvec{q})-\angle\myvec{q})} = \sum_{m=-\infty}^\infty J_m(|\myvec{p}|)J_{k-m}(|\myvec{q}|)e^{im(\angle\myvec{p}-\angle\myvec{q})}
\]
so
\begin{align*}
J_k(|\myvec{p}+\myvec{q}|)e^{ik\psi} & = \sum_{m=-\infty}^\infty J_m(|\myvec{p}|)J_{k-m}(|\myvec{q}|)e^{im(\phi-\pi)} \\
                                 & = \sum_{m=-\infty}^\infty J_m(|\myvec{p}|)J_{k-m}(|\myvec{q}|)(-1)^me^{im\phi}
\end{align*}

This is essentially Graf's addition theorem AS-9.1.79.

\section{A representation of $E(2)$}
We can now state a representation of $E(2)$.
Our carrier space is the space of ``Laurent'' series $\sum_{k=-\infty}^\infty a_kt^k$.
The representation is given by:
\begin{align*}
r(R_\phi) & : p(t) \rightarrow p(e^{i\phi}t)\\
r(T_{x,y}) & : p(t) \rightarrow e^{\frac{1}{2}(x(t-t^{-1})+iy(t+t^{-1}))}p(t) \\
r(C) & : p(t) \rightarrow p(-t^{-1})
\end{align*}

So in summary, we sought a basis of functions that is well behaved under rotation and translation.
We were ultimately led to the representation of $E(2)$ above.
And all of the properties of Bessel functions of the first kind (for integer modulus) follow from the fact that this is a true group representation, in particular the additions laws.

\end{document}
