\documentclass[12pt,reqno]{article}      % Sets 12pt font, equation numbers on right
\usepackage{amsmath,amssymb,amsfonts,amsthm} % Typical maths resource packages
\usepackage{graphics}                 % Packages to allow inclusion of graphics
\usepackage{color}                    % For creating coloured text and background
\usepackage{optidef}
\usepackage{bm}
\usepackage{framed}

\usepackage[colorlinks,citecolor=blue,linkcolor=blue]{hyperref}                 % For creating hyperlinks in cross references. It should be after the color package. The option colorlinks produces colored entries without boxes. The option citecolor=blue changes the default green citations to blue.

\theoremstyle{definition}
\newtheorem{example}{Example}
\newcommand{\dom}{\mathop{\textrm{dom}}}

\newcommand{\bmx}{\bm{x}}
\newcommand{\bmy}{\bm{y}}
\newcommand{\bma}{\bm{a}}
\newcommand{\bmb}{\bm{b}}
\newcommand{\bmc}{\bm{c}}
\newcommand{\bmA}{\bm{A}}
\newcommand{\bmlambda}{\bm{\lambda}}
\newcommand{\bmmu}{\bm{\mu}}
\newcommand{\bmnu}{\bm{\nu}}

\title{100 Optimization Problems and their Duals}

\begin{document}
\maketitle

\section{The Examples}

\begin{example}
This is the grandaddy of them all.
Every subsequent example is a special case of this one.

The primal problem:
\begin{mini}{x \in \cal{D}}{f(x)}{}{}
\addConstraint{g_i(x)}{\le 0,}{i=1,\ldots m}
\addConstraint{h_i(x)}{= 0,\quad}{i=1,\ldots n}
\end{mini}
\end{example}

We define the Lagrangian function by:
\[
L(x,\lambda,\mu) = f(x)+\sum_{i=1}^m\lambda_ig_i(x)+\sum_{i=1}^n\nu_ih_i(x)
\]

The dual function is now defined by
\[
g(\lambda,\nu) = \mathop{\inf}_{x\in\cal{d}}L(x,\lambda,\nu)
\]

The $g(\lambda,\nu)$ are all lower bounds for the minimum of the primal problem.

Define the dual problem as:
\begin{maxi}{\lambda,\nu}{g(\lambda,\nu)}{}{}
\addConstraint{\lambda}{\ge0}
\end{maxi}

The solution to the dual problem clearly bounds the solution to the primal problem from below.

Often there are values of $\lambda$ and $\nu$ for which $g(\lambda,\nu)=-\infty$ making $g(\lambda,\nu)$ vacuously a lower bound for the primal problem.
Define $\dom g$ as the set of $(\lambda,\nu)$ where $g(\lambda,\nu) > -\infty$.
Then the following is essentially the same same as the dual problem:
\begin{maxi}{\lambda,\nu}{g(\lambda,\nu)}{}{}
\addConstraint{\lambda}{\ge0}
\addConstraint{(\lambda,\nu)}{\in\dom g}
\end{maxi}
We will usually use the latter formulation.

We'll start with some problems that are simple enough that you might think of them as degenerate, but which still have non-trivial content.
\begin{example}
\begin{mini}{x \in \mathbb{R}}{f(x)}{}{}
\addConstraint{x=a}
\end{mini}
Clearly the solution is $f(a)$ achieved at $x=a$.

Define
\[
L(x,\nu) = f(x)+\lambda(a-x)
\]

We now have
\begin{align}
g(\lambda) & = \mathop{\inf}_{x}L(x,\lambda) \\
           & = \lambda a+\mathop{\inf}_{x}(f(x)-\lambda x) \\
\end{align}

The dual problem is
\begin{maxi}{\lambda \in \mathbb{R}}{g(\lambda)}{}{}
\end{maxi}
\end{example}

\subsubsection{Geometric interpretation}
Start with a line of the form $y = \lambda x+b$.
For a fixed $\lambda$, adjusting $b$ slides this line up and down the $y$-axis.
Consider the problem of sliding the line up so that it just touches the curve $y=f(x)$ from below.
We want to find $b$ such that the smallest gap between $y=f(x)$ and $y=\lambda x+b$ is zero.
If $b$ is the size of the smallest (signed) gap between $f(x)$ and $\lambda x$ then the line $y=\lambda x+b$ will be the translate of the line that just touches $y=f(x)$.
For any given $\lambda$, the line $y=\lambda(x-a)x+g(\lambda)$ is the line of gradient $\lambda$ that just touches $y=f(x)$.
Clearly $f(a) > \lambda(a-a)+g(\lambda) = g(\lambda)$.

Now consider the problem of finding $\lambda$ such the that the line just touches $y=f(x)$ from below at $x=a$.
This is clearly given by maximizing $g(]lambda)$.
So the dual problem finds gradient of a line just touching $y=f(x)$ from below.

Consider the case where $f(x)$ is differentiable.
Then the maximum in the dual problem occurs when $g'(\lambda) = 0$.
So we have $\lambda = f'(a)$.
In other words, the line with tangent $f'(a)$ just touches $f(x)$ from below at $x=a$.

\subsubsection{Physical interpretation}
Consider $f$ to be a potential field.
This gives rise to a force $-f'(x)$.
Suppose a particle is in the potential field.
What do we need to do to ensure the particle is constrained to $x=a$?
We need to supply a force $f'(a)$ that cancels out the force $-f'(x)$ due to the potential.
We can view this another way.
A fixed force $\lambda$ can be seen as the force arising from the potential $\lambda(a-x)$.
So to keep the particle static at $x=a$ we need to add a potential $\lambda(a-x)$ for some $\lambda$.
The particle is at rest at the minimum of the potential.
$g(λ)$ is the potential at $x=a$ when new potential is added.
Clearly it equals the minimum potential of a particle with a potential $f$ constrained to $a$.
This is generally true: we can can often think of the Lagrange multipliers as being the forces requireed to enforce a constraint.

\begin{example}
\begin{framed}
\begin{maxi}{\bmx\in\mathbb{R}^n}{\bmc\cdot \bmx}{}{}
\addConstraint{\bmA\bmx}{\le \bmb}
\addConstraint{\bmx}{\ge 0}
\end{maxi}
\end{framed}
\end{example}

The Lagrangian is given by
\begin{align}
L(\bmx,\bmlambda,\bmmu) &= \bmc\cdot\bmx+\bmlambda\cdot(\bmA\bmx-\bmb)+\bmmu\cdot x \\
                        &= (\bmc+\bmA^T\bmlambda+\bmmu)\cdot x-\bmb\cdot\bmlambda 
\end{align}
The maximum is trivially $\infty$ unless $\bmc+\bmA^T\bmlambda+\bmmu = 0$.
So we have
\[
g(\bmlambda,\bmmu) = -\bmb\cdot\bmlambda
\]
with $\dom g = \{(\lambda,\mu) \mid \bmc+\bmA^T\bmlambda+\bmmu = 0, \bmlambda \le 0, \bmmu \ge 0\}$.

So the dual problem is

\begin{mini}{\bmlambda,\bmmu\in\mathbb{R}^m}{-\bmb\cdot\bmlambda}{}{}
\addConstraint{\bmc+\bmA^T\bmlambda+\bmmu}{=0}
\addConstraint{\bmlambda}{\le0}
\addConstraint{\bmmu}{\ge0}
\end{mini}

In the presence of the condition $\bmmu\ge0$,  condition $\bmc+\bmA^T+\bmmu=0$ is equivalant to $\bmc+\bmlambda^T\bmA\le0$.
Define the new variable $\bmy=-\bmlambda$.
The dual problem is now
\begin{framed}
\begin{mini}{\bmy\in\mathbb{R}^m}{\bmb\cdot\bmy}{}{}
\addConstraint{\bmA^T\bmy}{\ge\bmc}
\addConstraint{\bmy}{\ge0}
\end{mini}
\end{framed}

\end{document}
