\documentclass[tikz]{article}
\usepackage{mathtools}
\usepackage{pgf}
\usepackage{tikz}
\usepackage{mathdots}
\usepackage{graphicx}
\usepackage{amsfonts}
\usetikzlibrary{arrows,automata,decorations.markings}
\makeatletter
\newcommand\mathcircled[1]{%
  \mathpalette\@mathcircled{#1}%
}
\newcommand\@mathcircled[2]{%
  \tikz[baseline=(math.base)] \node[draw,circle,inner sep=1pt] (math) {$\m@th#1#2$};%
}
\newcommand\myvec[1]{\vec{#1}\hspace{0.5mm}}
\makeatother
\newcommand\bbe{\mathbb{E}}
\newcommand\var{\mathop{\mathrm{Var}}}

% TODO
% More motivation with trig and elliptic examples
% More on C
% Better notation for T, R and C
% Some blurb about f(x,y) = f(r,\theta) = f(z) = f(\myvec{p})
% Some plots

\begin{document}
\section{A random point on a sphere}
Let $(X_1,X_2,\ldots,X_n)$ be a point chosen uniformly at random on an $(n-1)$-sphere.
What is the distribution of $X_1$?

By symmetry, we know that $\bbe[X_1]=0$.
We also know that $\bbe[X_1^2]+\ldots+\bbe[X_n^2]=1$, and that each of the $X_i$ has the same marginal distribution, and hence that $\bbe[X_1^2]=\frac{1}{n}$.
So $\var[X_1]=\frac{1}{n}$.
As $n$ grows, $X_1$ becomes more concentrated around zero.
In some sense, this means that random points on high-dimensional spheres are concentrated around the equator.
But this is somewhat illusory.
Random points on spheres are just as concentrated around any plane going through the origin.

Let's compute the exact distribution.
We start by computing the volume, $V_n(r)$, of an $n$-ball of radius $r$.
The volume of an $1$-ball of radius $r$, otherwise known as the interval $[-r, r]$, is $2r$.
We can slice an $n$-ball into $(n-1)$-balls at height $x$, radius $\sqrt{r^2-x^2}$ and thickness $dx$ which allows us to write
\begin{equation}
V_n(r) = \int dx V_{n-1}(\sqrt{r^2-x^2})
\label{slices}
\end{equation}

We can also view an $n$-ball as an onion of $(n-1)-spheres$, each of radius $r$ and thickness $dr$.
So
\[
V_n(r) = \int dr S_{n-1}(r)
\]
and hence that
\[
\frac{dV_n(r)}{dr} = S_{n-1}(r)
\]
Differentiating both sides of Equation~\ref{slices}, and sliding the derivative under the integral, gives
\[
S_{n-1}(r) = \int dx S_{n-2}(\sqrt{r^2-x^2})\frac{r}{\sqrt{r^2-x^2}}
\]
So our desired probability density function on $f$ is
\[
p(x) = \frac{S_{n-2}(\sqrt{r^2-x^2})}{S_{n-1}(r)}\frac{r}{\sqrt{r^2-x^2}}
\]
We also know that $S_n(r)=S_n(1)r^n$.
So $p(x) = A_n r^{-n+2}(r^2-x^2)^\frac{n-3}{2}$ for some constant $A_n$.

\begin{tikzpicture}
  \tikzset{swapaxes/.style = {rotate=90,yscale=-1}}
  \draw[red] plot[samples=200,domain=-2:2] function {(1-x*x/20)**8.5};
  \draw[->,thick] (-2,0)--(2,0) node[right]{$x$};
  \draw[->,thick] (0,-2)--(0,2) node[above]{$y$};
\end{tikzpicture}

\begin{tikzpicture}
  \tikzset{swapaxes/.style = {rotate=90,yscale=-1}}
  \draw[red] plot[samples=200,domain=-2:2] function {(1-x*x/40)**18.5};
  \draw[->,thick] (-2,0)--(2,0) node[right]{$x$};
  \draw[->,thick] (0,-2)--(0,2) node[above]{$y$};
\end{tikzpicture}
\end{document}
