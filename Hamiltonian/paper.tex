%&latex
\documentclass{article}
\usepackage{optidef}
\usepackage{calrsfs}
\usepackage{hyperref}

\begin{document}

%+Title
\title{Starting from Lagrangian mechanics, how to we get to Hamiltonian mechanics?}
\author{Dan Piponi}
\date{\today}
\maketitle
%-Title

%+Abstract
\begin{abstract}
Trying to motivate Hamilton's equations.
\end{abstract}
%-Abstract

%+Contents
\tableofcontents
%-Contents

\newcommand{\cD}{\mathcal{D}}
\newcommand{\rd}{\mathrm{d}}

\section{The Euler-Lagrange equations}
\subsection{Hamilton's principle of least action}
I'll start with the actual problem we want to solve.
We have a function of two variables, $L$, called the \emph{Lagrangian} and we want to find a function $q$ on some time interval (say $0$ to $T$) that minimises the so-called \emph{action}:
\[
A(q)=\int\rd tL(q,\dot{q}).
\]
Thinking of $q(t)$ parameterising a path then we are trying to find the path $q$ that minimises this action.
Many physical systems can be described as following a path $q$ that minimises the action for some definition of $L$.
This is Hamilton's principle of least action\cite{hamilton}.
A better name is Hamilton's princple of stationary action as the action might be a minimum, maximum of a saddle point.
But for now we'll work on the assumption that the Lagrangian is strictly convex and that we're looking for minima.

\subsection{A discretised version}
\newcommand{\Min}{\mathop{\mathrm{Min}}}
\newcommand{\Max}{\mathop{\mathrm{Max}}}

\newcommand{\bA}{\mathbf{A}}
I'm going to take the slightly non-standard approach of (1) generalising this problem slightly and (2) discretising it.
First the generalisation. We're going to minimise
\[
\int\rd tL(q,Aq)
\]
where $A$ is a linear operator acting on our space of paths.
Ultimately we're interested in the case when $A=\frac{\rd}{\rd t}$.
Next I'll discretise this problem as
\begin{equation*}
\min_{q_1}\min_{q_2}\ldots\min_{q_n}\sum_{i=1}^nL(q_i,(\bA q)_i)
\end{equation*}
Instead of $q$ depending continuously on $q$ we have $q_i$ for discrete time steps $i$.
And instead of the linear operator $A$ we have the matrix $\bA=(A_{ij})$ with $(\bA q)_i=\sum_jA_{ij}q_j$.
We'll use the shorthand
\begin{equation*}
\Min_q=\min_{q_1}\min_{q_2}\ldots\min_{q_n}
\end{equation*}
Our problem is now
\begin{equation}
\Min_q\sum_{i=1}^nL(q_i,(\bA q)_i)
\label{original}
\end{equation}

\subsection{The direct approach}
We can use standard calculus to attack (\ref{original}).
Differentiate with respect to $q_i$ and set the result equal to zero.
Define the vector $L^0=(L_i)$ by
\[
L^0_i=L(q_i,(\bA q)_i)
\]
and similarly $L^1$ and $L^2$ by
\begin{align*}
L^1_i &= L^{(1,0)}(q_i,(\bA q)_i)\\
L^2_i &= L^{(0,1)}(q_i,(\bA q)_i)\\
\end{align*}
where I use $f^{(m,n)}$ to mean $f$ differentiated $m$ times w.r.t.\ its first argument and $n$ times w.r.t.\ its second.
\newcommand{\pd}[1]{\frac{\partial}{\partial #1}}
The chain rule and a bit of index chasing gives
\begin{align*}
&\pd{q_k}\sum_{i=1}^nL(q_i,\sum_{j=1}^nA_{ij}q_j)=L^1_k+(\bA^TL^2)_k_ _\\
\end{align*}
so we need to solve
\[
L^1+A^TL^2=0
\]

\subsection{Back to the continuum}
When we go back to the continuum $q_i$ goes back to being $q(t)$ and the matrix $\bA$ goes back to being the linear operator $A$.
The transpose $\bA ^T$ becomes the adjoint $A^\ast$
defined by the property
\[
\int f(t)(Ag)(t)dx=\int (A^\ast f)(t)g(t)dt
\]
We need to solve the differential equation
\[
L^{(1,0)}(q,Aq)+A^\ast L^{(0,1)}(q,Aq)=0
\]
In the special case when $A=\frac{\rd}{\rd t}$ it is conventional to write this as
\[
\frac{\partial L}{\partial q}-\frac{d}{\rd t}\frac{\partial L}{\partial \dot{q}}=0
\]
which is known as the Euler-Lagrange equation.

\subsection{Why did I do things slightly non-standardly?}
\begin{enumerate}
\item What does $\pd{\dot{q}}$ mean? That's one of the most confusing bits of notation I've ever seen. The expression $\dot{q}$ doesn't name a variable but you normally differentiate w.r.t\ variables.
It looks like some kind of iterated derivative operation.
But really it just means the derivative with respect to the argument that goes in the slot currently occupied by $\dot{q}$ in the rest of the expression.
I find it easier to work with discrete indices first so there's no confusion over what is a function of what. 
\item I wanted to make it very clear where that minus sign in the Euler-Lagrange equation comes from. It's there because (a) the adjoint of differentiation is negated differentiation, (b) the adjoint is just a continuous version of the transpose and (c) the transpose is really just a bit of index shuffling that arises when you differentiate $x\cdot\bA y$ with respect to $y$.
\item In my day job I often find myself differentiating with respect to large sets of variables, but I rarely have anything like in calculus of variations where you effectively differentiate with respect to an infinite number of variables indexed by a continuous variable.
It's easier for me to trust what I'm familiar with.
\end{enumerate}

\section{Version 2: using an auxiliary variable}
\subsection{The easy case}

Consider the following minimisation
\begin{equation}
\min_{q_1}\min_{q_2}\ldots\min_{q_n}\sum_{i=1}^nL(q_i)
\label{simplified}
\end{equation}
We could imagine that $q_i$ is some quantity we need to decide daily for each day $i$, and that $L(q_i)$ is some kind of cost associated with $q_i$.
We're trying to minimise the total cost.
Because the quantity being optimised is a sum of terms, each of which depends on only one of the $q_i$, it becomes $n$ independent optimisations.
This makes life much easier.

\subsection{Back to the original problem}
Here's the original discretised problem again
\begin{equation}
\Min_q\sum_{i=1}^nL(q_i,(\bA q)_i)
\end{equation}
Using the example above, $\bA$ mixes up the $q_i$ for different days.
For example $L$ might depend on a moving average of $q$ for a few days or on the change in $q$ from one day to the next.
The problem no longer decomposes into $n$ independent problems.
One practical consequence of this is that (\ref{simplified}) can be implemented in constant time on an $n$ processor parallel machine, but (\ref{original}) doesn't seem amenable to such a treatment.

One way to handle this is to introduce an auxiliary variable $y$ and consider
the minimisation
\[
\Min_q\Min_y\sum_{i=1}^nL(q_i,y_i)\mbox{ subject to the constraint }y=\bA q
\]
Introduce a Lagrange multiplier $p$ and we instead consider
\[
\Min_q\Min_y\left(\sum_{i=1}^nL(q_i,y_i)+p\cdot(\bA q-y)\right)
\]
Let's assume everything in sight is strongly convex.
Then strong duality tells us that the original optimum is given by
\begin{align*}
&=\Max_p\Min_q\Min_y\left(\sum_{i=1}^nL(q_i,y_i)+p\cdot(\bA q-y)\right)\\
&=\Max_p\Min_q\left(\Min_y\left(\sum_{i=1}^n(L(q_i,y_i)-p_iy_i)\right)+p\cdot\bA q\right)\\
\end{align*}
We've partly succeeded in our goal: we have a subexpression that is the sum over values of $L$ at different $i$ meaning that they can be minimised independently.
An approach like this can turn into a practical algorithm.
For example we can take an alternating projection approach where we run the optimisations for the $y_i$ on each processor in parallel and then run the outer, but hopefully simpler and faster optimisations, in a more sequential manner.
If you follow this kind of idea you'll eventually reach something practical like ADMM \cite{boyd}. But curiously this idea also leads us to the Hamiltonian formulation of mechanics.

\subsection{The Hamiltonian}
Let's define the Hamiltonian $H$ by
\[
H(q,p)=\max_y(py-L(q,y))
\]
(Note that this is the Legendre transform of $H$ w.r.t.\ the second variable.)
Our original problem has become
\begin{equation}
\Max_p\Min_q\left(-\sum_iH(p_i,q_i)+p\cdot\bA q\right)
\label{reformulated}
\end{equation}
(In the special (\emph{separable}) case $L(q,y)=T(y)-V(q)$ where $T(y)=\frac{1}{2}y^2$.
Then
\begin{align*}
H(q,p)&=\max_y(py-\frac{1}{2}y^2+V(q))\\
&=\max_y(-\frac{1}{2}(y-p)^2+\frac{1}{2}p^2+V(q))\\
&=T(p)+V(q)\\
\end{align*}
which is often the total energy of the system.)

Now use some calculus to minimise our reformulated problem~\ref{reformulated}.
Assume everything s differentiable and that the quantity being optimized is convex with respect to $q$ and concave with respect to $p$ \cite{bogosel}:
\begin{align*}
\frac{\partial}{\partial p_i}(-H(q_i,p_i)+p_i(\bA q)_i)&=0\\
\frac{\partial}{\partial q_i}(-H(q_i,p_i)+p_i(\bA q)_i)&=0\\
\end{align*}

\[
(\bA q)_i=\frac{\partial H(q_i,p_i)}{\partial p_i}
\]
and
\[
(\bA^T p)_i=\frac{\partial H(q_i,p_i)}{\partial q_i}
\]

\subsection{Back to the continuous version}
Remember the original problem was
\[
\Min_q\int L(q,Aq)dt
\]
where we're now minimising over all functions $q$ and $A$ is some linear operator acting on $q$.
Translating sums to integrals and transposes to adjoints we get
\[
(Aq)(t)=\frac{\partial H(q,p)}{\partial p}
\]
and
\[
(A^\ast q)(t)=\frac{\partial H(q,p)}{\partial q}
\]
where $A$ is the adjoint of $A$ defined by
\[
\int f(t)(Ag)(t)dx=\int (A^\ast f)(t)g(t)dt
\]
In the special case that $A=\frac{d}{dt}$ we use integration by parts to find $A^\ast=-A$.
We end up with
\begin{align*}
\frac{dq}{dt}&=\phantom{-}\frac{\partial H}{\partial p}\\
\frac{dp}{dt}&=-\frac{\partial H}{\partial p}\\
\end{align*}
otherwise known as Hamilton's equations.
I hadn't noticed before but the different minus signs in the two equations arise from $A$ and its adjoint.

\subsection{TL;DR}
The Hamiltonian formulation of mechanics is a dual of the Lagrangian formulation. But it's not the direct dual, it's the dual after you introduce a new variable that is constrained to be equal to $\dot{q}$.

\section{Bonus: The quantum version}
We have the equality of (\ref{original}) and (\ref{reformulated}):
\begin{equation}
\Min_q\sum_{i=1}^nL(q_i,(\bA q)_i)
=\Max_p\Min_q\left(-\sum_iH(p_i,q_i)+p\cdot\bA q\right)
\end{equation}

Using the Legendre-Fourier dictionary \cite{piponi} all of the above reasoning can be reinterpreted.
We use the translation
\begin{align*}
\min_x \mbox{ and } \max_x &\rightarrow \int \rd x\\
\Max_x \mbox{ and } \Min_x &\rightarrow \int \cD x\\
f(x) &\rightarrow \exp(if(x))
\end{align*}
We get
\[
\int\cD q\exp i{\int \mathrm{d}tL(q,Aq)}=\int\cD q\cD p\exp i\int \mathrm{d}t( pAq-H(q,p)})
\]
and when $A=\frac{d}{dt}$ we get
\[
\int\cD q\exp i{\int \mathrm{d}tL(q,\dot{q})}=\int\cD p\cD q\exp i\int \mathrm{d}t(
p\dot{q}-H(q,p)})
\]
Now compare with Ryder \cite{ryder} Chapter 5 eqs. (5.13) and (5.15).
This one of the important steps in Feynman's derivation of the path integral formulation.
%+Bibliography
\begin{thebibliography}{99}
\bibitem{hamilton} \url{https://en.wikipedia.org/wiki/Principle_of_least_action}
\bibitem{ryder} Ryder, Lewis H., \emph{Quantum Field Theory}, Cambridge University Press
\bibitem{boyd} \url{http://stanford.edu/~boyd/admm.html}
\bibitem{bogosel} Beni Bogosel \url{https://math.stackexchange.com/users/7327/beni-bogosel)}
\bibitem{piponi} Me \url{https://github.com/dpiponi/Legendre/blob/master/legendre.pdf}
\end{thebibliography}
%-Bibliography

\end{document}


